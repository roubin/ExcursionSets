\documentclass[a4paper,12pt]{article}

%% SHORTCUTS
\input{shortcuts_excursion_set}

%% PACKAGES
%\usepackage{amsfonts}
\usepackage{amsmath} % for math fonts
\usepackage{mathrsfs} % for \mathscr
\usepackage{amsfonts} % for \mathbb


\usepackage{graphicx}
\usepackage{subfigure}

%% TITLES
\title{Handbook on excursion sets}
\author{Emmanuel Roubin}
\date{Version: \today}


\begin{document}
\maketitle
\tableofcontents
\newpage
\section{Priliminary calculations}
\subsection{Differentiation under the integral sign}
\begin{equation}
  \frac{\text{d}}{\text{d}x}\left(\ \int_{a(x)}^{b(x)} f(x,t) dt \ \right) = f(x,b(x))b'(x)-f(x,a(x))a'(x)+\int_{a(x)}^{b(x)}\frac{\partial}{\partial x}f(x,t)dt
\end{equation}
\subsection{Functions}
\subsubsection{Gamma function: $\GamFunc$}
\begin{equation}\GamFunc: z \rightarrow \int_0^\infty t^{z-1}e^{-t} dt\end{equation}
\subsubsection{Error function: $\erf$}
\begin{equation}\erf(x)=\frac{2}{\sqrt\pi}\int_0^xe^{-t^2}dt\end{equation}
\subsection{Lipschitz-Killing curvatures (LKC)}
\subsubsection{For a cube in 1,2 and 3 dimensions}
\begin{equation}
  \text{For} \ \ \MRF_N=\prod_{i=1}^{N}[0\ \Msize], \ \ \LKC_j(\MRF_N)= \left( \begin{array}{c} N \\ j \end{array} \right) \Msize^j
\end{equation}
\begin{center}
  \begin{tabular}{c|c|c|l}
    $N$ & LKC & Value & Meaning \\
    \hline
    $1$ & $\LKC_0(\MRF)$ & $1$ & Euler Characteristic \\
    $1$ & $\LKC_1(\MRF)$ & $\Msize$ & Length of the segment \\
    \hline
    $2$ & $\LKC_0(\MRF)$ & $1$ & Euler Characteristic \\
    $2$ & $\LKC_1(\MRF)$ & $2\Msize$ & Half the boundary length \\
    $2$ & $\LKC_2(\MRF)$ & $\Msize^2$ & Surface area \\
    \hline
    $3$ & $\LKC_0(\MRF)$ & $1$ & Euler Characteristic \\
    $3$ & $\LKC_1(\MRF)$ & $3\Msize$ & Twice the caliper diameter \\
    $3$ & $\LKC_2(\MRF)$ & $3\Msize^2$ & Half the surface area \\
    $3$ & $\LKC_3(\MRF)$ & $\Msize^3$ & Volume \\
  \end{tabular}
\end{center}

\subsection{Volume of the unit ball}
\begin{minipage}{0.5\linewidth}
\begin{equation}\UB_N=\frac{\pi^{N/2}}{\GamFunc(1+N/2)}\end{equation}
\end{minipage}
\begin{minipage}{0.5\linewidth}
\begin{equation}
  \left|
  \begin{array}{l}
    \UB_0=1 \\
    \UB_1=2 \\
    \UB_2=\pi \\
    \UB_3=4\pi/3 \\
    \dots
  \end{array}
  \right.
\end{equation}
\end{minipage}

\subsection{Probabilistic Hermite polynomials}
\begin{minipage}{0.5\linewidth}
\begin{equation}\Hermite_n=(-1)^n e^{x^2/2} \frac{\text{d}^n}{\text{d}x^n}e{-x^2/2}\end{equation}
\end{minipage}
\begin{minipage}{0.5\linewidth}
\begin{equation}
  \left|
  \begin{array}{l}
    \Hermite_0(x)=1 \\
    \Hermite_1(x)=x \\
    \Hermite_2(x)=x^2-1 \\
    \Hermite_3(x)=x^3-3 \\
    \dots
  \end{array}
  \right.
\end{equation}
\end{minipage}

\subsection{Flag coefficients}
\begin{minipage}{0.5\linewidth}
\begin{equation}
  \left[ \begin{array}{c} n \\ j \end{array} \right] = 
  \left( \begin{array}{c} n \\ j \end{array} \right)\frac{\UB_n}{\UB_{n-j}\UB_j}
\end{equation}
\end{minipage}
\begin{minipage}{0.5\linewidth}
\begin{equation}
  \begin{array}{c|c|c}
    n & j & \text{flag} \\
    \hline
    0 & 0 & 1\\
    1 & 0 & 1\\
    1 & 1 & 1\\
    2 & 0 & 1\\
    2 & 1 & \pi/2\\
    2 & 2 & 1\\
    3 & 0 & 1\\
    3 & 1 & 2\\
    3 & 2 & 2\\
    3 & 3 & 1
  \end{array}
\end{equation}
\end{minipage}

\subsection{Second spectral moment}
\begin{equation}
  \SpecMom=\Varia\left\{\frac{\partial\, \GRF(\x)}{\partial\, x_i}\right\}= \frac{\text{d}^2\cov(h)}{\text{d}h^2}
\end{equation}
\begin{equation}
  \begin{array}{c|c|c}
    \text{Covariance} & \text{Parameter} & \SpecMom \\
    \hline
    \text{Gaussian} & - & 2\std^2/\Lc^2\\
  \end{array}
\end{equation}

\section{Gaussian Minkowsky functionals}
\subsection{Preliminary}
\subsubsection{Gaussian measure: $\GaussMeas^k(\HS)$}
\begin{equation}
  \GaussMeas^k(\HS) = \Proba\{\vGRV\in\HS\}=\frac{1}{\std^k(2\pi)^{k/2}} \int_\HS e^{-\|\x-\mean\|^2/2\std^2} d\x
\end{equation}
\subsubsection{Tube Taylor expansion}
\begin{equation}
  \GaussMeas^k(\tube(\HS,\ray))=\sum_{j=0}^{\infty} \frac{\ray^j}{j!}\GMFk_j(\HS)
\end{equation}

\subsection{GMFs for Gaussian distribution}
\subsubsection{Application to one-dimensional tail hitting set: $k=1$ and $\HS=\uinf$}
Tail probability $\TP$:
\begin{equation}
  \GaussMeas^1(\uinf) = \Proba\{\GRV\ge\lset\}=\frac{1}{\std\sqrt{2\pi}} \int_\lset^\infty e^{-(x-\mean)^2/2\std^2} dx=\TP(\lset)
\end{equation}
Error function $\erf$:
\begin{subequations}
  \begin{align*}
    \GaussMeas^1(\uinf) = \TP(\lset) & = \frac{1}{\std\sqrt{2\pi}} \int_{\frac{\lset-\mean}{\std\sqrt{2}}}^\infty e^{-t^2} \sqrt{2}\std dt = \frac{1}{\sqrt{\pi}} \int_{\frac{\lset-\mean}{\std\sqrt{2}}}^\infty e^{-t^2} dt \\
                          & = \frac{1}{\sqrt{\pi}} \left[ \int_0^\infty e^{-t^2} dt - \int_0^{\frac{\lset-\mean}{\std\sqrt{2}}} e^{-t^2} dt \right]\\
                          & = \frac{1}{\sqrt{\pi}} \left[ \frac{\sqrt{\pi}}{2} - \frac{\sqrt{\pi}}{2}\erf\left(\frac{\lset-\mean}{\std \sqrt{2}}\right) \right]\\
                          & = \frac{1}{2}\left(1-\erf\left(\frac{\lset-\mean}{\std\sqrt{2}}\right)\right)
  \end{align*}
\end{subequations}
Tube expansion:
\begin{equation}
  \GaussMeas^1(\tube(\uinf,\ray))=\GaussMeas^1([\lset-\ray\ \infty))=\TP(\lset-\ray)=\sum_{j=0}^{\infty} \frac{(-\ray)^j}{j!} \frac{\text{d}^j \TP(\lset)}{\text{d} \lset^j}
\end{equation}
Identification:
\begin{itemize}
\item Notations:
  \begin{subequations}
    \begin{flalign}
      & \text{Tail hitting set:} \ \ \HST=\uinf & \\
      & \text{Variable substitution:} \ \ \clset \leftarrow \frac{\lset-\mean}{\std} &
    \end{flalign}
  \end{subequations}
\item For $j=0$:
\begin{equation}
  \GMFa_0(\HST)=\TP(\lset)
\end{equation}
\item For $j>0$:
\begin{equation}
  \GMFa_j(\HST)=(-1)^j\frac{\text{d}^j\TP(\lset)}{\text{d}\lset^j}=\frac{e^{-(\lset-\mean)^2/2\std^2}}{\std^j\sqrt{2\pi}}\Hermite_{j-1}((\lset-\mean)/\std)
\end{equation}
\item First values:
\begin{equation}
  \left|
  \begin{array}{l}
    \GMFa_0(\HST) = \frac{1}{2}\left(1-\erf\left(\frac{\clset}{\sqrt{2}}\right)\right) \\ \\
    \GMFa_1(\HST) = \frac{1}{\std  \sqrt{2\pi}}\ e^{-\clset^2/2} \\ \\
    \GMFa_2(\HST) = \frac{1}{\std^2\sqrt{2\pi}}\ \clset \ e^{-\clset^2/2} \\ \\
    \GMFa_3(\HST) = \frac{1}{\std^3\sqrt{2\pi}}\ \left(\clset^2-1\right) \ e^{-\clset^2/2} \\ \\
    \GMFa_4(\HST) = \frac{1}{\std^4\sqrt{2\pi}}\ \left(\clset^3-3\clset\right) \ e^{-\clset^2/2} \\ \\
    \dots
  \end{array}
  \right.
\end{equation}
%\begin{figure}[!h]
%  \scalebox{0.9}{\input{./figures/gmf_tail}}
%  \caption{First 4 GMF for $\std^2=2$ and $\HS=\HST=\uinf$}
%\end{figure}
\item Derivatives:
\begin{equation}
  \forall j, \ \ \frac{\text{d}\GMFa_j}{\text{d}\lset}(\HST)=(-1)^j\frac{\text{d}^{j+1}\TP(\lset)}{\text{d}\lset^{j+1}}=-\GMFa_{j+1}(\HST)
\end{equation}
\end{itemize}

\subsubsection{Application to one-dimensional cumulative hitting set: $k=1$ and $\HS=\minfu$}
Cumulative probability $\CDP$:
\begin{equation}
  \GaussMeas^1(\minfu) = \Proba\{\GRV\le\lset\}=\frac{1}{\std\sqrt{2\pi}} \int_{-\infty}^\lset e^{-(x-\mean)^2/2\std^2} dx=\CDP(\lset)
\end{equation}
Error function $\erf$:
\begin{equation}
  \GaussMeas^1(\minfu)=\CDP(\lset)=\frac{1}{2}\left(1+\erf\left(\frac{\lset-\mean}{\std\sqrt{2}}\right)\right)
\end{equation}
Tube expension:
\begin{equation}
  \GaussMeas^1(\tube(\minfu,\ray))=\GaussMeas^1((-\infty\ \lset+\ray])=\CDP(\lset+\ray)=\sum_{j=0}^{\infty} \frac{\ray^j}{j!} \frac{\text{d}^j \CDP(\lset)}{\text{d} \lset^j}
\end{equation}
Identification:
\begin{itemize}
\item Notations:
  \begin{subequations}
    \begin{flalign}
      & \text{Tail hitting set:} \ \ \HSC=\minfu & \\
      & \text{Variable substitution:} \ \ \clset \leftarrow \frac{\lset-\mean}{\std} &
    \end{flalign}
  \end{subequations}
\item For $j=0$:
\begin{equation}
  \GMFa_0(\HSC)=\CDP(\lset)
\end{equation}
\item For $j>0$:
\begin{equation}
  \GMFa_j(\HSC)=\frac{\text{d}^j\CDP(\lset)}{\text{d}\lset^j}=(-1)^{j+1}\frac{e^{-(\lset-\mean)^2/2\std^2}}{\std^j\sqrt{2\pi}}\Hermite_{j-1}((\lset-\mean)/\std)
\end{equation}
\item First values:
\begin{equation}
  \left|
  \begin{array}{l}
    \GMFa_0(\HSC) = \frac{1}{2}\left(1-\erf\left(\frac{\clset}{\sqrt{2}}\right)\right) \\ \\
    \GMFa_1(\HSC) = \frac{1}{\std  \sqrt{2\pi}}\ e^{-\clset^2/2} \\ \\
    \GMFa_2(\HSC) = \frac{-1}{\std^2\sqrt{2\pi}}\ \clset \ e^{-\clset^2/2} \\ \\
    \GMFa_3(\HSC) = \frac{1}{\std^3\sqrt{2\pi}}\ \left(\clset^2-1\right) \ e^{-\clset^2/2} \\ \\
    \GMFa_4(\HSC) = \frac{-1}{\std^4\sqrt{2\pi}}\ \left(\clset^3-3\clset\right) \ e^{-\clset^2/2} \\ \\
    \dots
  \end{array}
  \right.
\end{equation}
\begin{figure}[!h]
  \centering
  \subfigure[$\HS=\HSC=\uinf$]{\scalebox{0.5}{\input{./figures/gmf_tail}}}\hfill
  \subfigure[$\HS=\HSC=\minfu$]{\scalebox{0.5}{\input{./figures/gmf_cumulative}}}
  \caption{First 4 GMF for $\std^2=2$}
\end{figure}
\item Derivatives:
\begin{equation}
  \forall j, \ \ \frac{\text{d}\GMFa_j}{\text{d}\lset}(\HSC)=\frac{\text{d}^{j+1}\CDP(\lset)}{\text{d}\lset^{j+1}}=\GMFa_{j+1}(\HSC)
\end{equation}
\end{itemize}

\subsection{GMFs for Gaussian related Random Fields}
\begin{equation}
  \GRRF : \Univ\times\RN \overset{\vGRF}{\rightarrow} \Rk \overset{\rfunc}{\rightarrow} \RR
\end{equation}
\begin{equation}
    \Proba\left\{\GRRF\in\HS\right\} = \Proba\left\{\rfunc(\GRF)\in\HS\right\} = \Proba\left\{\GRF\in\irfunc(\HS)\right\} = \GaussMeas^k(\irfunc(\HS))
\end{equation}

\subsubsection{Log-normal distribution}
\begin{equation}
  k=1 \ \ \text{,} \ \ \rfunc=\exp \ \ \text{and} \ \ \irfunc=\ln
\end{equation}
\begin{equation}
  \left|
  \begin{array}{lcl}
    \mean & = &\ln(\meanln) - \frac{1}{2}\ln(1+\stdln^2/\meanln^2) \\
    \std^2 & = & \ln(1+\stdln^2/\meanln^2)
  \end{array}
  \right.
\end{equation}

\begin{itemize}
\item Notations:
  \begin{subequations}
    \begin{flalign}
      & \text{Log tail hitting set:} \ \ \lnHST=\lnuinf & \\
      & \text{Log cumulative hitting set:} \ \ \lnHSC = \minflnu & \\
      & \text{Variable substitution:} \ \ \lnclset \leftarrow \frac{\ln(\lset)-\mean}{\std} &
    \end{flalign}
  \end{subequations}
\item For $\HS=\HST=\uinf \ \ \rightarrow \ \ \irfunc(\HS) = \lnHST = \lnuinf$
\begin{equation}
  \left|
  \begin{array}{l}
    \GMFa_0(\lnHST) = \frac{1}{2}\left(1-\erf\left(\frac{\lnclset}{\sqrt{2}}\right)\right) \\ \\
    \GMFa_1(\lnHST) = \frac{1}{\std  \sqrt{2\pi}}\ e^{-\lnclset^2/2} \\ \\
    \GMFa_2(\lnHST) = \frac{1}{\std^2\sqrt{2\pi}}\ \lnclset \ e^{-\lnclset^2/2} \\ \\
    \GMFa_3(\lnHST) = \frac{1}{\std^3\sqrt{2\pi}}\ \left(\lnclset^2-1\right) \ e^{-\lnclset^2/2} \\ \\
    \GMFa_4(\lnHST) = \frac{1}{\std^4\sqrt{2\pi}}\ \left(\lnclset^3-3\lnclset\right) \ e^{-\lnclset^2/2} \\ \\
    \dots
  \end{array}
  \right.
\end{equation}
Derivatives:
\begin{equation}
  \left|
  \begin{array}{l}
    \text{Help:} \ \ \left(e^{-(\ln(x)-\mean)^2/2\std^2}\right)' = -\frac{1}{\std^2} \frac{\ln(x)-\mean}{x}\ e^{-(\ln(x)-\mean)^2/2\std^2} \\ \\
    \frac{\text{d} \GMFa_0(\lnHST)}{\text{d}\lset} = \frac{-1}{\std\sqrt{2\pi}} \frac{1}{\lset}\ e^{-(\ln(\lset)-\mean)^2/2\std^2} \\ \\
    \frac{\text{d} \GMFa_1(\lnHST)}{\text{d}\lset} = \frac{-1}{\std^2\sqrt{2\pi}} \frac{1}{\lset}\frac{\ln(\lset)-\mean}{\std}\ e^{-(\ln(\lset)-\mean)^2/2\std^2} \\ \\
    \frac{\text{d} \GMFa_2(\lnHST)}{\text{d}\lset} = \frac{-1}{\std^3\sqrt{2\pi}} \frac{1}{\lset}\left[ \frac{(\ln(\lset)-\mean)^2}{\std^2} - 1 \right]\ e^{-(\ln(\lset)-\mean)^2/2\std^2} \\ \\
    \frac{\text{d} \GMFa_3(\lnHST)}{\text{d}\lset} = \frac{-1}{\std^4\sqrt{2\pi}} \frac{1}{\lset}\left[ \frac{(\ln(\lset)-\mean)^3}{\std^3} - 3\frac{(\ln(\lset)-\mean)}{\std} \right]\ e^{-(\ln(\lset)-\mean)^2/2\std^2} \\ \\
    \dots \\ \\
    \text{Guess would be:} \ \ \forall j\ge0, \\
    \frac{\text{d} \GMFa_j(\lnHST)}{\text{d}\lset} = \frac{-1}{\std^{j+1}\sqrt{2\pi}}\frac{1}{\lset}\Hermite_{j}(\lnclset)\ e^{-\lnclset^2/2}
  \end{array}
  \right.
\end{equation}

\item For $\HS=\zu \ \ \rightarrow \ \ \irfunc(\HS) = \lnHSC = \minflnu$
\begin{equation}
  \left|
  \begin{array}{l}
    \GMFa_0(\lnHSC) = \frac{1}{2}\left(1+\erf\left(\frac{\lnclset}{\sqrt{2}}\right)\right) \\ \\
    \GMFa_1(\lnHSC) = \frac{1}{\std  \sqrt{2\pi}}\ e^{-\lnclset^2/2} \\ \\
    \GMFa_2(\lnHSC) = \frac{-1}{\std^2\sqrt{2\pi}}\ \lnclset \ e^{-\lnclset^2/2} \\ \\
    \GMFa_3(\lnHSC) = \frac{1}{\std^3\sqrt{2\pi}}\ \left(\lnclset^2-1\right) \ e^{-\lnclset^2/2} \\ \\
    \GMFa_4(\lnHSC) = \frac{-1}{\std^4\sqrt{2\pi}}\ \left(\lnclset^3-3\lnclset\right) \ e^{-\lnclset^2/2} \\ \\
    \dots
  \end{array}
  \right.
\end{equation}
Derivatives:
\begin{equation}
  \left|
  \begin{array}{l}
    \text{Guess would be:} \ \ \forall j\ge0, \\
    \frac{\text{d} \GMFa_j(\lnHSC)}{\text{d}\lset} = \frac{(-1)^{j}}{\std^{j+1}\sqrt{2\pi}}\frac{1}{\lset}\Hermite_{j}(\lnclset)\ e^{-\lnclset^2/2}
  \end{array}
  \right.
\end{equation}

\end{itemize}

\newpage
\section{Expectation Formula}
\subsection{Gaussian distribution}
\subsubsection{General case}
\begin{equation}
  \Expec\left\{\ \LKC_j(\ES) \ \right\} = \sum_{i=0}^{N-j} \left[ \begin{array}{c}i+j\\i\end{array}\right]\left(\frac{\SpecMom}{2\pi}\right)^{i/2}\LKC_{i+j}(\MRF)\GMFk_i(\irfunc(\HS))
\end{equation}
Notations:
  \begin{subequations}
    \begin{flalign}
      & \text{LKC of $\MRF$:} \ \ \LKC_j(\MRF) = \LKC_j^\MRF & \\
      & \text{LKC of $\ES$:} \ \ \LKC_j(\ES) = \LKC_j^\ES & \\
      & \text{GMF of $\irfunc(\HS)$:} \ \  \GMFk_j(\irfunc(\HS)) = \GMFk_j
    \end{flalign}
  \end{subequations}
\subsubsection{One dimension}
\begin{subequations}
\begin{align}
  &\Expec\left\{\LKC_0^\ES\right\} = \left(\frac{\SpecMom}{2\pi}\right)^{1/2}\LKC_1^\MRF\GMFk_1 + \LKC_0^\MRF\GMFk_0\\
  &\Expec\left\{\LKC_1^\ES\right\} = \LKC_1^\MRF\GMFk_0
\end{align}
\end{subequations}


\begin{figure}[!h]
  \centering
  \subfigure[Euler Characteristic]{\scalebox{0.5}{\input{./figures/elkc_1D_LKC0}}}\hfill
  \subfigure[Length]{\scalebox{0.5}{\input{./figures/elkc_1D_LKC1}}} 
  \caption{Expectation of LKC for $\Msize=100$, $\std^2=2$ and $\HS=\uinf$ in 1D.}
\end{figure}

\begin{figure}[!h]
  \centering
  \subfigure{\scalebox{0.5}{\input{./figures/elkc_exp_1D_lognormal_0}}}\hfill
  \subfigure{\scalebox{0.5}{\input{./figures/elkc_exp_1D_lognormal_1}}} 
  \caption{Expectation of LKC for $\Msize=100$, $\mean=0.5$, $\std^2=2$, $\Lc=0.5$, and $\HS=\zu$ (log-normal distribution with cumulative hitting set) in 1D. Experimental results are calculated over $1\,000$ realizations.}
\end{figure}

\newpage 
\subsubsection{Two dimensions}
\begin{subequations}
\begin{align}
  &\Expec\left\{\LKC_0^\ES\right\} = \frac{\SpecMom}{2\pi}\LKC_2^\MRF\GMFk_2 + \sqrt{\frac{\SpecMom}{2\pi}}\LKC_1^\MRF\GMFk_1 + \LKC_0^\MRF\GMFk_0\\
  &\Expec\left\{\LKC_1^\ES\right\} = \sqrt{\frac{\SpecMom\pi}{8}}\LKC_2^\MRF\GMFk_1 + \LKC_1^\MRF\GMFk_0\\
  &\Expec\left\{\LKC_2^\ES\right\} = \LKC_2^\MRF\GMFk_0
\end{align}
\end{subequations}


\begin{figure}[!h]
  \centering
  \subfigure{\scalebox{0.35}{\input{./figures/elkc_exp_2D_gaussian_0}}}\hfill
  \subfigure{\scalebox{0.35}{\input{./figures/elkc_exp_2D_gaussian_1}}}\hfill
  \subfigure{\scalebox{0.35}{\input{./figures/elkc_exp_2D_gaussian_2}}} 
  \caption{Expectation of LKC for $\Msize=100$, $\mean=0.0$, $\std^2=2$, $\Lc=5$, and $\HS=\uinf$ (gaussian distribution with tail hitting set) in 2D. Experimental results are calculated over $1\,000$ realizations.}
\end{figure}

\begin{figure}[!h]
  \centering
  \subfigure{\scalebox{0.35}{\input{./figures/elkc_exp_2D_lognormal_0}}}\hfill
  \subfigure{\scalebox{0.35}{\input{./figures/elkc_exp_2D_lognormal_1}}}\hfill
  \subfigure{\scalebox{0.35}{\input{./figures/elkc_exp_2D_lognormal_2}}} 
  \caption{Expectation of LKC for $\Msize=100$, $\mean=0.0$, $\std^2=2$, $\Lc=5$, and $\HS=\zu$ (lognormal distribution with cumulative hitting set) in 2D. Experimental results are calculated over $1\,000$ realizations.}
\end{figure}

\newpage
\subsubsection{Three dimensions}
\begin{subequations}
\begin{align}
  &\Expec\left\{\LKC_0^\ES\right\} = \\%\frac{\SpecMom}{2\pi}\LKC_2^\MRF\GMFk_2 + \sqrt{\frac{\SpecMom}{2\pi}}\LKC_1^\MRF\GMFk_1 + \LKC_0^\MRF\GMFk_0\\
  &\Expec\left\{\LKC_1^\ES\right\} = \\%\sqrt{\frac{\SpecMom\pi}{8}}\LKC_2^\MRF\GMFk_1 + \LKC_1^\MRF\GMFk_0\\
  &\Expec\left\{\LKC_2^\ES\right\} = \\%\LKC_2^\MRF\GMFk_0
  &\Expec\left\{\LKC_3^\ES\right\} = \\%\LKC_2^\MRF\GMFk_0
\end{align}
\end{subequations}

\begin{figure}[!h]
  \centering
  \subfigure{\scalebox{0.35}{\input{./figures/elkc_exp_3D_gaussian_0}}}\hspace{1.2cm}
  \subfigure{\scalebox{0.35}{\input{./figures/elkc_exp_3D_gaussian_1}}}\\
  \subfigure{\scalebox{0.35}{\input{./figures/elkc_exp_3D_gaussian_2}}}\hspace{1.2cm}
  \subfigure{\scalebox{0.35}{\input{./figures/elkc_exp_3D_gaussian_3}}}\\
  \caption{Expectation of LKC for $\Msize=100\times 20 \times 20$, $\mean=0.0$, $\std^2=2$, $\Lc=2$, and $\HS=\uinf$ (gaussian distribution with tail hitting set) in 3D. Experimental results are calculated over $100$ realizations.}
\end{figure}

\newpage
\section{Size effect}
\subsection{One dimensional case}
\begin{figure}[!h]
  \centering
  \subfigure[Variation of the quantile]{\scalebox{0.35}{\input{./figures/size_effect_failure_stress_m10_v10}}}\hfill
  \subfigure[Variation of the variance]{\scalebox{0.35}{\input{./figures/size_effect_failure_stress_m10_q001}}} \hfill
  \subfigure[Variation of the mean]{\scalebox{0.35}{\input{./figures/size_effect_failure_stress_v10_q001}}}
  \caption{Size Effect: variations around $\meanln=10$, $\stdln^2=10$ and $q=1\%$ in 1D.}
\end{figure}


\subsection{Two dimensional case}
\subsubsection{Behavior of the Euler characteristic for different specimen sizes}
\begin{figure}[!h]
  \centering
  \subfigure[$\Msize=0.001$\label{fig:2D:EC:small_size}]{\scalebox{0.35}{\input{./figures/size_effect_2D_different_Euler_behaviors_l0.001}}}\hfill
%  \subfigure[$\Msize=0.8$]  {\scalebox{0.5}{\input{./figures/size_effect_2D_different_Euler_behaviors_l0.8}}}\\
  \subfigure[$\Msize=2.5$\label{fig:2D:EC:medium_size}]  {\scalebox{0.35}{\input{./figures/size_effect_2D_different_Euler_behaviors_l2.5}}}\hfill
  \subfigure[$\Msize=8$\label{fig:2D:EC:large_size}]    {\scalebox{0.35}{% GNUPLOT: LaTeX picture with Postscript
\begingroup
  \makeatletter
  \providecommand\color[2][]{%
    \GenericError{(gnuplot) \space\space\space\@spaces}{%
      Package color not loaded in conjunction with
      terminal option `colourtext'%
    }{See the gnuplot documentation for explanation.%
    }{Either use 'blacktext' in gnuplot or load the package
      color.sty in LaTeX.}%
    \renewcommand\color[2][]{}%
  }%
  \providecommand\includegraphics[2][]{%
    \GenericError{(gnuplot) \space\space\space\@spaces}{%
      Package graphicx or graphics not loaded%
    }{See the gnuplot documentation for explanation.%
    }{The gnuplot epslatex terminal needs graphicx.sty or graphics.sty.}%
    \renewcommand\includegraphics[2][]{}%
  }%
  \providecommand\rotatebox[2]{#2}%
  \@ifundefined{ifGPcolor}{%
    \newif\ifGPcolor
    \GPcolorfalse
  }{}%
  \@ifundefined{ifGPblacktext}{%
    \newif\ifGPblacktext
    \GPblacktexttrue
  }{}%
  % define a \g@addto@macro without @ in the name:
  \let\gplgaddtomacro\g@addto@macro
  % define empty templates for all commands taking text:
  \gdef\gplbacktext{}%
  \gdef\gplfronttext{}%
  \makeatother
  \ifGPblacktext
    % no textcolor at all
    \def\colorrgb#1{}%
    \def\colorgray#1{}%
  \else
    % gray or color?
    \ifGPcolor
      \def\colorrgb#1{\color[rgb]{#1}}%
      \def\colorgray#1{\color[gray]{#1}}%
      \expandafter\def\csname LTw\endcsname{\color{white}}%
      \expandafter\def\csname LTb\endcsname{\color{black}}%
      \expandafter\def\csname LTa\endcsname{\color{black}}%
      \expandafter\def\csname LT0\endcsname{\color[rgb]{1,0,0}}%
      \expandafter\def\csname LT1\endcsname{\color[rgb]{0,1,0}}%
      \expandafter\def\csname LT2\endcsname{\color[rgb]{0,0,1}}%
      \expandafter\def\csname LT3\endcsname{\color[rgb]{1,0,1}}%
      \expandafter\def\csname LT4\endcsname{\color[rgb]{0,1,1}}%
      \expandafter\def\csname LT5\endcsname{\color[rgb]{1,1,0}}%
      \expandafter\def\csname LT6\endcsname{\color[rgb]{0,0,0}}%
      \expandafter\def\csname LT7\endcsname{\color[rgb]{1,0.3,0}}%
      \expandafter\def\csname LT8\endcsname{\color[rgb]{0.5,0.5,0.5}}%
    \else
      % gray
      \def\colorrgb#1{\color{black}}%
      \def\colorgray#1{\color[gray]{#1}}%
      \expandafter\def\csname LTw\endcsname{\color{white}}%
      \expandafter\def\csname LTb\endcsname{\color{black}}%
      \expandafter\def\csname LTa\endcsname{\color{black}}%
      \expandafter\def\csname LT0\endcsname{\color{black}}%
      \expandafter\def\csname LT1\endcsname{\color{black}}%
      \expandafter\def\csname LT2\endcsname{\color{black}}%
      \expandafter\def\csname LT3\endcsname{\color{black}}%
      \expandafter\def\csname LT4\endcsname{\color{black}}%
      \expandafter\def\csname LT5\endcsname{\color{black}}%
      \expandafter\def\csname LT6\endcsname{\color{black}}%
      \expandafter\def\csname LT7\endcsname{\color{black}}%
      \expandafter\def\csname LT8\endcsname{\color{black}}%
    \fi
  \fi
  \setlength{\unitlength}{0.0500bp}%
  \begin{picture}(7200.00,5040.00)%
    \gplgaddtomacro\gplbacktext{%
      \csname LTb\endcsname%
      \put(682,704){\makebox(0,0)[r]{\strut{}-3}}%
      \put(682,1074){\makebox(0,0)[r]{\strut{}-2}}%
      \put(682,1444){\makebox(0,0)[r]{\strut{}-1}}%
      \put(682,1814){\makebox(0,0)[r]{\strut{} 0}}%
      \put(682,2184){\makebox(0,0)[r]{\strut{} 1}}%
      \put(682,2554){\makebox(0,0)[r]{\strut{} 2}}%
      \put(682,2925){\makebox(0,0)[r]{\strut{} 3}}%
      \put(682,3295){\makebox(0,0)[r]{\strut{} 4}}%
      \put(682,3665){\makebox(0,0)[r]{\strut{} 5}}%
      \put(682,4035){\makebox(0,0)[r]{\strut{} 6}}%
      \put(682,4405){\makebox(0,0)[r]{\strut{} 7}}%
      \put(682,4775){\makebox(0,0)[r]{\strut{} 8}}%
      \put(814,484){\makebox(0,0){\strut{} 0.001}}%
      \put(2012,484){\makebox(0,0){\strut{} 0.01}}%
      \put(3210,484){\makebox(0,0){\strut{} 0.1}}%
      \put(4407,484){\makebox(0,0){\strut{} 1}}%
      \put(5605,484){\makebox(0,0){\strut{} 10}}%
      \put(6803,484){\makebox(0,0){\strut{} 100}}%
      \csname LTb\endcsname%
      \put(176,2739){\rotatebox{90}{\makebox(0,0){\strut{}Euler characteristic}}}%
      \put(3808,154){\makebox(0,0){\strut{}$\lset$}}%
    }%
    \gplgaddtomacro\gplfronttext{%
    }%
    \gplbacktext
    \put(0,0){\includegraphics{./figures/size_effect_2D_different_Euler_behaviors_l8.eps}}%
    \gplfronttext
  \end{picture}%
\endgroup
}}
  \caption{Euler characteristic for different specimen sizes $\Msize$ for $\meanln=0.5$, $\stdln^2=2$, $\Lc=1$, and $\HS=\zu$ (lognormal distribution with cumulative hitting set) in a 2D.\label{fig:2D:EC}}
\end{figure}

\begin{itemize}
  \item [\ref{fig:2D:EC:small_size}] Small sizes \\ \textbf{Direct percolation}. The first (and only) $\lset(\EC==q)$ correspond to the probability $q$ of percolation. There is only one disconnected element in the excursion set.
  \item [\ref{fig:2D:EC:medium_size}] Medium sizes \\ 
  \item [\ref{fig:2D:EC:large_size}] Large sizes \\ 
\end{itemize}


\end{document}
